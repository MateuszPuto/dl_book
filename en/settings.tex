% !TeX spellcheck = en_US
\usepackage[utf8]{inputenc} % kodowanie polskich znaków
\usepackage[T1]{fontenc}

\usepackage{textcomp} %TRADEMARKI  I INNE SYMBOLE
\usepackage{verbatim,amsmath,mathtools,amsfonts,amssymb,amsthm,bm}
\usepackage[polish,english]{babel}

%% Links in pdf
\usepackage{color}
\definecolor{linkcol}{rgb}{0,0,0.4}
\definecolor{citecol}{rgb}{0.5,0,0}
\usepackage[a4paper,hyperindex=true]{hyperref} %pagebackref

\hypersetup
{
bookmarksopen=true,
pdftitle="Notatnik Sztucznej Inteligencji",
pdfauthor="Michał Dyzma", 
pdfsubject="Notatnik Sztucznej Inteligencji", %subject of the document
%pdftoolbar=false, % toolbar hidden
pdfmenubar=true, %menubar shown
pdfhighlight=/O, %effect of clicking on a link
colorlinks=true, %couleurs sur les liens hypertextes
%pdfpagemode=None, %aucun mode de page
%pdfpagelayout=SinglePage, %ouverture en simple page
pdffitwindow=true, %pages ouvertes entierement dans toute la fenetre
linkcolor=linkcol, %couleur des liens hypertextes internes
citecolor=citecol, %couleur des liens pour les citations
urlcolor=linkcol %couleur des liens pour les url
}

%%Wyglad pdf 
\usepackage{url}
\usepackage{color}
\usepackage{fix-cm}
\usepackage{setspace}  %odstepy miedzy wierszami
\usepackage{ae,aecompl}
\usepackage{lipsum} %generowanie losowego tekstu
\usepackage{titling}
\usepackage{xcolor}
\usepackage[version=3]{mhchem}
\usepackage{indentfirst}
\usepackage{lscape} %poziome strony
\usepackage{latexsym}
\setcounter{MaxMatrixCols}{10}
%\usepackage{menukeys} %Sciezki dostepu z ikonami katalogow oraz ladne sekwencje klawiszy
\usepackage{eqnarray}



% Dwie kolumny w spisie treści
%\usepackage[toc]{multitoc}

%%Fancy chapter
\usepackage{tikz}

\usepackage{titlesec, blindtext}
%\definecolor{yourcolor}{HTML}{008bb2}

%styl wyswietlania tytulu rozdzialu
\makechapterstyle{mystyle}{%
  \chapterstyle{default}
  \renewcommand*{\chapnumfont}{\normalfont\Huge\sffamily\bfseries}
  \renewcommand*{\chaptitlefont}{\normalfont\huge\rmfamily\color{black}\bfseries}
  \settowidth{\chapindent}{\chapnumfont 111}
  \renewcommand*{\chapterheadstart}{}
  \renewcommand*{\chapnamefont}{\hfill\color{black}\normalfont\Huge\rmfamily\bfseries}
  \renewcommand*{\printchapternum}{%
  \begin{tikzpicture}[baseline={([yshift=-.6ex]current bounding box.center)}]
  \node[fill=black,circle,text=white] {\thechapter};
  \end{tikzpicture}\\[1ex]
  \hrule height 1.5pt}
  \renewcommand*{\printchaptertitle}[1]{%
    {\chaptitlefont ##1}}
}

%use new chapter style
%\chapterstyle{mystyle}
%\definecolor{gray75}{gray}{0.75}
%\newcommand{\hsp}{\hspace{20pt}}
%\titleformat{\chapter}[hang]{\Huge\bfseries}{\thechapter\hsp\textcolor{gray75}{|}\hsp}{0pt}{\Huge\bfseries}
%--------------
%\makechapterstyle{box}{
%  \renewcommand*{\printchaptername}{}
%  \renewcommand*{\chapnumfont}{\normalfont\sffamily\huge\bfseries}
%  \renewcommand*{\printchapternum}{
%    \flushright
%    \begin{tikzpicture}
%      \draw[fill,color=black] (0,0) rectangle (2cm,2cm);
%      \draw[color=white] (1cm,1cm) node { \chapnumfont\thechapter };
%    \end{tikzpicture}
%  }
%  \renewcommand*{\chaptitlefont}{\normalfont\sffamily\Huge\bfseries}
%  \renewcommand*{\printchaptertitle}[1]{\flushleft\chaptitlefont##1}
%}
%\chapterstyle{box}

%%...tabel
%\usepackage{tabu} %colorowe czcionki i linie w tabelach
\usepackage{rotating} %obracabie tabel na stronie
\usepackage{array}
\newcolumntype{P}[1]{>{\raggedright\arraybackslash}p{#1}}
\usepackage{hhline}
\usepackage{multicol,multirow}
\usepackage{booktabs,bigdelim,bigstrut}
\usepackage{colortbl}
\usepackage{longtable}

\definecolor{LightCyan}{rgb}{0.88,1,1}
\definecolor{LimeGreen}{rgb}{0.19,0.83,0.19}

%%...obrazkow
\usepackage{graphicx}
\usepackage{rotating}
\usepackage{float}
\usepackage[section]{placeins} %kontroluje położenie float-ów
\usepackage{wrapfig}
\usepackage[labelsep=quad,indention=10pt]{subfig}
\captionsetup*[subfigure]{position=bottom}

%%Formatowanie czcionek podpisow, tytułów rozdziałów etc
%\usepackage[font=footnotesize,labelfont=bf,format=default,justification=centerlast]{caption}
\usepackage{titlesec} %zmiana czcionek w tytule i sekcjach


%%Listy, numerowania, numeracja wierszy
\usepackage{enumerate}
\usepackage{enumitem}
\usepackage{lineno}  % numerowanie wierszy
\usepackage{listings}
\usepackage{verbatim}


%%Cytowania
%\usepackage{cite}
%\usepackage[nonamebreak,comma,sort,compress,square,numbers]{natbib}



%%Tikz
\usepackage{tikz}
\usetikzlibrary{arrows,positioning,shapes,calc,matrix,shadows}


%%Skroty i definicje, skorowidz
\usepackage[polish,intoc]{nomencl}
\makenomenclature



% % % % % % % % % % % % % % % % % % % % % % % % % % % % % % % % % % % % % % % % % % % %
% % % % % % % % % % % %  Definicje  % % % % % % % % % % % % % % % % % % % % % % % % % %

%%Podwojne puste strony
\makeatletter
\def\cleardoublepage{\clearpage\if@twoside
\ifodd\c@page
\else\hbox{}\thispagestyle{empty}\newpage
\if@twocolumn \hbox {}\newpage \fi \fi \fi }
\makeatother

%% Styl punktowania

\newenvironment{bulletList}%
{ \begin{list}%
	{$\bullet$}%
	{\setlength{\labelwidth}{25pt}%
	 \setlength{\leftmargin}{50pt}%
	 \setlength{\itemsep}{\parsep}}}%
{ \end{list}}


\usepackage{geometry}
%\geometry{
%a4paper,
%total={210mm,297mm},
%left=20mm,
%right=20mm,
%top=20mm,
%bottom=20mm,
%}

%\makeatletter
%\renewcommand*{\num@bs}{\section{\abstractname}}
%\makeatother

%****************Definicje Bogdana****************
\def \p {\partial}

\def \B   {\hfill $\Box$}



\def \eps {\varepsilon}

%\def \beq {\begin{equation}}
%
%\def \enq  {\end{equation}}

\def \w {\widetilde}

\def \d {\dfrac} 

\newtheorem{lemma}{Lemat}
\newtheorem{defin}{Definicja}
% \newtheorem{lemmas}{Lemma}
\newtheorem{assum}{{\textbf{Założenie}}}
\newtheorem{thm}{\textbf{Twierdzenie}}
\newtheorem{propos}{\textbf{Propozycja}}
%\captionnamefont{\textbf{\small}}
%\captiontitlefont{\textbf{\small}}

%***************************************************

\maxsecnumdepth{subsection}
\maxtocdepth{subsubsection}
\setlength\columnsep{.8cm}

\chapterstyle{mystyle}

\captionnamefont{\bfseries}
\captiontitlefont{\bfseries}
%\addto\captionspolish{\renewcommand{\figurename}{Rycina}}


% dzielenie wierszy
\widowpenalty=10000
\clubpenalty=10000 
\brokenpenalty=10000
\raggedbottom


%%  Margins
\setlrmarginsandblock{3cm}{2.5cm}{*}
\setulmarginsandblock{2cm}{2.5cm}{*}
\checkandfixthelayout
% % % % % % % % % % % % % % % % % % % % % % % % % % % % % % % % % % % % % % % % % % %
% % % % % % % % % % % % % % % % % % % % % % % % % % % % % % % % % % % % % % % % % % %


\makeatletter
%\renewcommand*{\l@chapter}[2]{%
%	\l@chapapp{\makebox[7em][l]{\hfill\chaptername~}#1}{#2}{\cftchaptername}}
\renewcommand*{\l@appendix}[2]{%
	\l@chapapp{\makebox[4em][l]{\hfill\appendixname~}#1}{#2}{\cftappendixname}}
\makeatother