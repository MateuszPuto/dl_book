%-*- root:../main.tex -*-
\setcounter{page}{1}
%\begin{bulletList}
%\item First point
%\item Second point
%\item Here is an abbreviation reference \nomenclature{DTI}{Diffusion Tensor Imaging} DTI
%\end{bulletList}
%
\renewcommand{\chaptername}{Rozdział}
\chapter{Wstęp}
\label{chap:intro}


\section{Cele...}


Cel jest intuicyjnie zdefiniowany jako pragnienie, życzenie lub coś w kierunku czego dążymy. W codziennym życiu często mówimy o naszych celach. Są one tym, wokół czego skupiają się nasze myśli i rozmowy. Dzieje się tak, ponieważ cele ze względu na to, że nie możemy ich zrealizować od razu, są związane z pewnym komponentem czasowym. To znaczy, że nasze cele nie zmieniają się wraz z upływem czasu. Dokładniej mówiąc, cele nadrzędne nie ulegają zmianie, podczas gdy cele podrzędne, służące realizacji tych ważniejszych dla nas celów, są dostosowywane do zmieniającej się sytuacji. Mówimy np. że naszym celem jest zrobienie lub osiągnięcie czegoś, co sądzimy, że przyniesie nam pewną korzyść, lecz często nie sama korzyść jest tym, czego pragniemy, a raczej dążenie do zrealizowania celu nadrzędnego, który ma dla nas ponadprzeciętną wartość. Ze względu na długotrwałość i stosunkową rzadkość występowania, pewne cele są nam bliskie, jesteśmy z nimi zżyci i utożsamiamy się z nimi. Mimo tego istnieje też rodzaj celów, których używamy instrumentalnie. Te cele służą nam zazwyczaj tylko przez krótki czas, występują częściej i są jedynie wykonawcami celów nadrzędnych. Tak samo, jak my mamy rozbudowane relacje i metody radzenia sobie z celami, w podobny sposób powinno się to odbywać w dziedzinie sztucznej inteligencji. Jednak, ze względu na ogromne skomplikowanie naszych metod wyszukiwania i rozwiązywania problemów, próba imitacji ludzkiego zachowania wydaje się ponad nasze siły. Pomyślmy o tym, że przecież nawet dokładnie nie wiemy jakie procesy biorą udział w naszym własnym rozumowaniu. Czy w chwili kiedy mamy moment ‘Aha’ wiemy jakie procesy nas do niego doprowadziły? Także nauka pozostawia nas z ograniczonymi metodami badania mózgu z powodu jego odizolowania w czaszce. Pomimo wprowadzenia nowych metod obrazowania, opracowywanych coraz to nowych leków działających na mózg, czy operacji na nim, wciąż niewiele wiadomo o sposobach jego działania. Nie jesteśmy także jeszcze pewni, w jaki sposób przełożyć to, co wiemy na algorytmy, a to jest coś, co będzie nas najbardziej interesować w kontekście dziedziny sztucznej inteligencji. W konsekwencji nie pozostaje nam nic innego jak pójść drogą matematyki i spróbować nieco sformalizować znaczenia słowa „cel”. Tak samo, jak my mamy, często burzliwe, relacje z naszymi celami, tak samo powinniśmy podać osobę, rzecz czy byt, który będzie odpowiedzialny za realizowanie zadanego mu przez nas celu. Ten byt będzie w specjalnej relacji ze swoim celem, to znaczy połączymy go z nim raz na zawsze, tak aby w każdej chwili był skupiony na swoim celu. Takie połączenie przypomina małżeństwo, i tak jak małżeństwo taką relację można zerwać, lecz nie obejdzie się to bez pewnych perturbacji. Może się okazać że jeśli zamienimy cel, to nasz byt nie będzie go realizował tak samo dobrze. Może się też okazać, że będzie go realizować w sposób lepszy niż losowy, co będziemy mogli wykorzystać. Nie wnikając teraz jednak w szczegóły, skupimy się na definicji. Wykonawcę, inaczej mówiąc, istotę, która posiada pewien cel, nazwiemy zgodnie z literaturą RL, \textbf{aktorem} (ang. agent). Dalej, aby pokazać rzeczywiste działanie prezentowanych systemów, będziemy wprowadzać co jakiś czas definicje. Powiedzieliśmy, że do realizowania celu jest nam potrzebny aktor, jak go nazwaliśmy. Na razie posługiwaliśmy się mglistą definicją samego celu. Teraz podamy bardziej formalną definicję: Cel określimy jako wartość numeryczną, która jest dostępna dla aktora w pewnych momentach czasu. Aktor chce również, aby ta wartość była jak największa. Jest to jednoznaczne z matematyczną definicją maksymalizacji, więc możemy zapisać, że:

\begin{equation}
v = max(z)
\end{equation}

\noindent Gdzie $\boldsymbol{v}$ jest celem, a $\boldsymbol{z}$ jest zmienną wartością, która zależy od pewnych znanych lub nieznanych czynników. $\boldsymbol{V}$ nazwiemy funkcją wartości.\newline

\noindent \textbf{Funkcja wartości} (ang. value function) jest pewną funkcją, która określa, w jakim stopniu aktor zrealizował swój cel. Mówi nam ona, ile nagrody otrzyma on w wyniku swoich działań i jak dobrze radzi on sobie w danej sytuacji, przypisując jakąś liczbę, określającą jak bardzo dana sytuacja jest pożądaną, do każdej możliwej sytuacji. Oczywiście funkcja wartości może się zmieniać wraz z upływem czasu i ta zmiana będzie nam mówić o poprawie lub pogorszeniu sytuacji aktora. O celu możemy równocześnie myśleć intuicyjnie tak, jak w codziennym życiu. Tu celem może być bogactwo, miłość czy też wiedza, a odpowiadającą tym celom funkcją wartości może być kolejno ilość pieniędzy na koncie, siła uczucia drugiej osoby i ilość przeczytanych książek. Dostęp do funkcji wartości możemy sobie wyobrazić jako odczuwanie, tak jak sami możemy odczuwać czy jest nam dobrze, czy źle, zimno czy ciepło itd.\newline

\noindent Podajmy jeszcze jeden przykład:\newline

\noindent Powiedzielibyśmy, że celem istoty w ujęciu darwinistycznym jest przetrwanie i reprodukcja. Tak więc spróbujmy to zapisać formalnie za pomocą naszej definicji.

\begin{equation}
v = max(p + k * r)
\end{equation}

\noindent Gdzie $\boldsymbol{p}$ jest zmienną związaną z przetrwaniem (wyrażoną np. w latach życia), $\boldsymbol{r}$  jest zmienną związaną z reprodukcją (np. liczba potomstwa) i ponieważ możemy cenić potomstwo bardziej lub mniej niż rok życia dodajemy kurs wymiany $\boldsymbol{k}$  między tymi dwoma zmiennymi. Równanie (1.2) pokazuje jeden z możliwych celów, którymi mogą kierować się aktorzy. W ciągu czytania tego tekstu poznamy różne inne cele, które mogą zostać użyte w innych okolicznościach.

\section{... i jak je osiągnąć}

Powiedzieliśmy, że realizatorem celów są aktorzy. Jednak ta definicja pomija najważniejsze pytanie, a mianowicie: Jak aktorzy realizują swoje cele? Pod tym problemem kryje się najwięcej trudności i to jest właśnie główny cel sztucznej inteligencji (albo krócej: AI od ang. artificial intelligence), jakim jest tworzenie aktorów, którzy osiągają jakiś cel, a więc maksymalizują jakąś zadaną wartość. Często ci, którzy tworzą takie rozwiązania nie myślą nawet o aktorach, bo rozumieją bez zastanawiania się, że potrzebują pewnego modelu rzeczywistości, który będzie wyjaśniał działanie systemu, którym się zajmują. Takim modelem będą właśnie procesy myślowe aktora. Tak więc główny nacisk jest położony na osiąganie jasno zdefiniowanych celów, czyli takich, które da się mierzyć, za pomocą komputerów. To bardzo ogólne podejście, za pomocą którego możemy próbować rozwiązać wszelakie ciężkie problemy. Jedynymi ograniczeniami są nasza zdolność zdefiniowania problemu i umiejętność przekazania rozwiązania w języku zrozumiałym dla komputerów. Oczywiście możemy też nie znać rozwiązania problemu, który zdefiniowaliśmy i wtedy dodatkową trudnością jest to, że nasz system musi być w stanie znaleźć zadowalające nas rozwiązanie. Znajdywanie tych rozwiązań, będzie najważniejszą częścią wiedzy o systemach sztucznej inteligencji. W pewnym więc sensie badanie AI jest nauką nakierowaną na rozwiązanie wszystkich istniejących problemów. W pierwszej chwili może wydawać się to nam myśleniem życzeniowym albo nawet, że jest to nieosiągalne, przecież jeśli istniałby prosty, definiowalny matematycznie sposób radzenia sobie z każdą napotykaną sytuacją to na pewno ktoś odkryłby go już dawno temu. Jednak co ciekawe, mimo iż pewne intuicje na temat działania mózgu były obecne od dawna, to próby systematycznego opisu jego działania rozpoczynają się w XIX w. w psychologii i na przełomie wieków w matematyce jako logika. W następującym potem okresie psychologia posłużyła jako inspiracja, do badania procesów występujących w mózgu a logika dostarczyła, poza systemem rozumowania, podstawy do stworzenia głównego obiektu do przeprowadzania eksperymentów, jakim okazał się komputer. Mimo iż rozwiązania tu opisane korzystają szczodrze z rozwiązań znanych w matematyce od wieków, to mogły one zostać zastosowane dzięki rozległej mocy obliczeniowej dostarczonej przez nowoczesne komputery. Nie oznacza to, jednak że wszystko było gotowe i czekało na zastosowanie. Chociażby sieci neuronowe, według naszej wiedzy, nie śniły się nikomu przed wynalezieniem komputerów. Jest to bardzo ciekawe gdyż model sieci neuronowej lub chociażby samego neuronu, zdaje się bardzo prosty, po jego zrozumieniu. Okazuje się, że nasza wyobraźnia mimo swojej mocy jest ograniczona, więc większość rozwiązań, o których tu mowa zostało opracowanych dopiero w ciągu kilkudziesięciu ostatnich lat. Właściwie, jak zobaczysz w ostatnim rozdziale, ten proces nadal trwa. Prawdopodobnie zostało nam jeszcze wiele do odkrycia, zanim będziemy mogli uznać problem za całkowicie rozwiązany. A jeśli chodzi o nadmierną łatwość takiego podejścia do AI, to jak to bywa, wiele rozwiązań, które mogą się wydawać łatwe na początku, okazuje się niezmiernie skomplikowane i trudne w wykonaniu. Chociaż dłuto może ociosać kamień w dowolnym kształcie, to stworzenie katedry za pomocą dłuta jest wielkim wyzwaniem. Zobaczymy, iż formalizowanie pewnych intuicyjnych koncepcji może być zawiłe i że dyscyplina sztucznej inteligencji jest w istocie dziedziną inżynieryjną.

\section{Właściwy wstęp}

Powiedzieliśmy już, czym zajmuje się dziedzina sztucznej inteligencji, jednak nie wspomnieliśmy, że bardzo podobną dziedziną wykorzystującą komputery do rozwiązywania problemów jest informatyka. Jaka jest różnica między dziedziną informatyki a dziedziną sztucznej inteligencji? Sztuczna inteligencja wywodzi się z informatyki na wiele sposobów. Zazwyczaj ci sami ludzie pracujący w wydziałach informatyki zajmowali się też problemami sztucznej inteligencji, używali do tego znanych przez siebie metod, czyli metod przetwarzania informacji. Czytelnik mógłby nawet powiedzieć, że rozwiązania, które kiedyś uważane były za AI, obecnie uchodzą za zwyczajne algorytmy. Jednak zwyczajową datą uznawaną za początek AI jest rok 1956, kiedy to odbył się tzw. warsztat w Dartmouth. W nocie proponującej ten warsztat napisano m.in. że proponujący warsztat, czyli McCarthy, Minsky, Rochester i Shannon, nazwiska ważne dla dziedziny, myślą, że dziesięć osób pracujących przez lato jest w stanie dokonać znaczących postępów. Pomimo wielkich planów efektem były raczej przyjaźnie i ekscytacja nowymi możliwościami niż realne postępy. Jak więc sami uczestnicy tego warsztatu definiują pojęcie AI? John McCarthy: [AI to] „nauka i inżynieria tworzenia inteligentnych maszyn”. Moglibyśmy podejść do sprawy w ten sam sposób, przyjmując, że AI jako dziedzina zajmuje się badaniem zagadnienia, jak stworzyć sztuczną wersję prawdziwej inteligencji, jak mogłaby to sugerować sama nazwa. Ale będziemy przeciwni takiej interpretacji, częściowo z powodu braku jasności takiego postawienia sprawy. W końcu czy na pewno wiemy, czym jest inteligencja? Oczywiście McCarthy działał prawdopodobnie celowo, proponując rozmytą definicję na początku istnienia dziedziny, tak aby nie była ona przeszkodą dla naukowców. Jednak obecnie, definicja, którą uważamy za lepszą, brzmi następująco:\newline

\noindent \textbf{Sztuczna inteligencja} to dziedzina wiedzy zajmująca się badaniem programów komputerowych, które w nietrywialny sposób używają wcześniej zdefiniowanych zasad, żeby tworzyć nowe niezdefiniowane wcześniej zasady, które pomagają w osiąganiu zadanych celów.\newline

Z definicji tej wynika, że aby program został uznany za sztuczną inteligencję, musi w znaczącym stopniu wykazywać się zmianą swoich decyzji w zależności od napotkanych sytuacji. Wyobraźmy sobie np. grę w kamień-papier-nożyce. Tworząc program grający w tę grę, moglibyśmy zaprogramować zasadę, że jeśli przeciwnik używa ponadprzeciętnej ilości „kamieni” to zawsze gramy w odpowiedzi „papier”, podobnie dla innych możliwości. Teraz, jeśli przeciwnik zacznie nadużywać którejś możliwości to program go za to ukara. Jest to jednak podejście proceduralne, wyliczające wszystkie możliwości. Alternatywą byłoby zastosowanie ogólnej zasady z kilkoma przypadkami. Możemy stworzyć zasadę: jeśli przeciwnik nadużywa $\boldsymbol{x}$ to graj $\boldsymbol{y}$ które pokonuje $\boldsymbol{x}$. Teraz wystarczy, że system podstawi odpowiednie wartości zgodnie z kolejnoścą: kamień, papier, nożyce. W ten sposób mamy uniwersalną zasadę z trzema przypdkami, a więc spełniamy definicję sztucznej inteligencji, jeśli uznamy to działanie za nietrywialne. Głównym problemem tej definicji jest właśnie pojawiające się tam słowo ‘nietrywialne’, które możemy rozumieć tylko intuicyjnie. W pewnym sensie jednak to, co będzie się kryć pod tym wyrazem, będzie związane z naszą wiedzą w tej dziedzinie.

\section{Jakie tematy poruszymy w tym kursie}

Podstawą zrozumienia tego kursu jest pewna znajomość matematyki. Jednak jest on tak pomyślany, aby osoba nieposiadająca części niezbędnej wiedzy, takiej jak np. podstawy różniczkowania, mogła zrozumieć jak najwięcej. Zaawansowane tematy nie będą zawarte w tym tekście i nie to było myślą autora. Raczej niż wprowadzać jak najwięcej tematów, autor jest przekonany, że lepiej jest posiąść głębokie zrozumienie podstaw niż mieć nikłe pojęcie o przeróżnych pomysłach, nie będąc w stanie ich odtworzyć. Autor chce zawrzeć w tym kursie matematyczną optymalizację, sieci neuronowe, drzewa poszukiwań oraz uczenie przez wzmacnianie. Dla osoby pierwszy raz stykającej się z tematem może się to wydać niedużą ilością materiału, ekspert może myśleć, że pokryjemy każdy temat źle. W umyśle autora wszystkie te obiekcje są ważne, ale tematy zostały wybrane ze względu na tworzenie jak najbardziej zawierającej się w sobie historii, która byłaby interesująca dla czytelnika i jednocześnie prezentowała najpotężniejsze istniejące techniki.
